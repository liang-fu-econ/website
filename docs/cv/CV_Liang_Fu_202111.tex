%%%%%%%%%%%%%%%%%%%%%%%%%%%%%%%%%%%%%%%%%
% Important note:
% This template requires the resume.cls file to be in the same directory as the .tex file. 
% The resume.cls file provides the resume style used for structuring the document.
%%%%%%%%%%%%%%%%%%%%%%%%%%%%%%%%%%%%%%%%%






%----------------------------------------------------------------------------------------
%	PACKAGES AND OTHER DOCUMENT CONFIGURATIONS
%----------------------------------------------------------------------------------------

\documentclass{resume_liang} % Use the custom resume.cls style

\usepackage[left=0.75in, top=0.6in, right=0.75in, bottom=0.6in]{geometry} % Document margins
\usepackage[colorlinks=true, citecolor=blue]{hyperref} % Hypertext links in the document
% citecolor (green): Color for bibliographical citations in text.
\usepackage{tabu}

\newcommand{\tab}[1]{\hspace{.2667\textwidth}\rlap{#1}}
\newcommand{\itab}[1]{\hspace{0em}\rlap{#1}}




\name{Liang Fu} % Your name
\address{\small Last updated: November 2021}
%\address{Hudson Building - Room 103} % Your address
%\address{1400 Washington Avenue, Albany, NY 12222} % Your secondary address (optional)
%\address{\href{mailto:lfu@albany.edu}{lfu@albany.edu} \\ (+1) 518-423-8555} %  Your phone number and your email






\begin{document}

	
%----------------------------------------------------------------------------------------
%	Contact Information
%----------------------------------------------------------------------------------------
\begin{tabu} to \linewidth {X[l]X[r]}
	%	 & \emph{Office:} (212) 998-8036 \\
Hudson Building, Room 103  & \href{mailto:lfu@albany.edu}{\color{blue}{lfu@albany.edu}} \\
1400 Washington Avenue     & +1 (518)423-8555 \\
Albany, NY 12222           & \href{https://liang-fu-econ.github.io/website/}{\color{blue}{https://liang-fu-econ.github.io/website/}}
\end{tabu}
\bigskip \bigskip
%----------------------------------------------------------------------------------------
%	Contact Information
%----------------------------------------------------------------------------------------
%\begin{rSection}{Contact Information}
%{\bf Email:} \href{mailto:lfu@albany.edu}{lfu@albany.edu}   \\
%{\bf Phone:} (+1) 518-423-8555  \\
%{\bf Office Address:} Hudson Building - Room 103, 1400 Washington Avenue, Albany, NY 12222 \\
%{\bf Home Address:} 101 Lincoln Avenue, Albany, NY 12206
%\end{rSection}
%\bigskip \bigskip




%----------------------------------------------------------------------------------------
%	Education 
%----------------------------------------------------------------------------------------
\begin{rSection}{Education}
%--copy and paste this region  if you need more--
Ph.D. Economics, {\bf \href{https://www.albany.edu/economics}{\color{black}{University at Albany, SUNY}}}  \hfill {} \vspace{0.1cm}              
  \begin{itemize}
  	\item[] \underline{Expected Completion}: May 2022
  	%\item[] \underline{Dissertation}: Three essays in monetary economics
  \end{itemize}
M.A. Economics, {\bf \href{http://en.cufe.edu.cn/}{\color{black}{Central University of Finance and Economics}}} \hfill {2014} 
\vspace{0.1cm} \\
B.S. Management, {\bf \href{https://english.qdu.edu.cn/}{\color{black}{Qingdao University}}} \hfill {2011} 
\vspace{0.1cm} \\
\end{rSection}
\bigskip  




%----------------------------------------------------------------------------------------
%	Research and Teaching Fields
%----------------------------------------------------------------------------------------
\begin{rSection}{Research and Teaching Fields}
{\bf Primary fields:} Monetary Economics, International Economics 
\vspace{0.1cm} \\ 
{\bf Secondary fields:} Macroeconomics, Chinese Economy
\\  
\end{rSection}
\bigskip  




%--------------------------------------------------------------------------------
%    Research Papers
%-----------------------------------------------------------------------------------------------
\begin{rSection}{Research Papers}
``Monetary Policy Surprises and Interest Rates under China's Evolving Monetary Policy Framework'' (with Chun-Yu Ho) \textbf{Job Market Paper}, Revision Requested by \textit{\color{blue}{\underline{\textbf{Emerging Markets Review}}}}.
\vspace{0.2cm} \\
``Political Stability and Credibility of Currency Boards'' (with Shu Feng, Chun-Yu Ho, and Wai-Yip Alex Ho) Revision Requested by \textit{\color{blue}{\underline{\textbf{Journal of International Money and Finance}}}}.
\vspace{0.2cm} \\
``Real Exchange Rate and Innovation: Firm-Level Evidence from China'' (with Chun-Yu Ho and Xiaoli Zhang) 
%\vspace{0.1cm} \\
%\textbf{``Differential Regional Impact of U.S. Monetary Policy: Evidence from a Near VAR''}
\end{rSection}
\bigskip  




%--------------------------------------------------------------------------------
%    Research Papers in Progress
%-----------------------------------------------------------------------------------------------
\begin{rSection}{Research Papers in Progress}
``The Impact of China's Monetary Policy on Exchange Rates: Evidence from High-Frequency Data''\vspace{0.2cm}
\\
``Time-Varying Effects of China's Monetary Policy: Evidence from Time-Varying Parameter VAR Model with Stochastic Volatility'' (with Cheng Yang)
\\
\end{rSection}
\bigskip  




%----------------------------------------------------------------------------------------
%    Teaching Experience
%----------------------------------------------------------------------------------------
\begin{rSection}{Teaching Experience}
{\bf Instructor}{, University at Albany, SNUY} \hfill { 08/2018 -- current}
 \begin{itemize}
 	 \item[] Economic Statistics (Undergraduate): Fall 2021
     \item[] Money and Banking (Undergraduate): Summer 2018, Fall 2018, Fall 2019, Spring 2020
     \item[] Intermediate Macroeconomics (Undergraduate): Spring 2019, Fall 2020, Spring 2021
 \end{itemize} 
{\bf Teaching Assistant}{, University at Albany, SUNY} \hfill {09/2014 -- 05/2016}
   \begin{itemize}
     \item[] Microeconomics II (PhD): Spring 2016
     \item[] Public Microeconomics (Undergraduate): Spring 2016
     \item[] Applied Econometrics (Undergraduate): Fall 2015
     \item[] Economics of Labor (Undergraduate): Fall 2014, Spring 2015
   \end{itemize} 
{\bf Teaching Assistant}{, Central University of Finance and Economics} \hfill {02/2012 -- 01/2013}
\begin{itemize}
	\item[] Production and Consumption Theory (Undergraduate): Fall 2012
	\item[] Intermediate Microeconomics (Undergraduate): Spring 2012 
\end{itemize} 
\vspace{0.2cm}
\end{rSection}
\bigskip  




%----------------------------------------------------------------------------------------
%	Research Experience 
%----------------------------------------------------------------------------------------
\begin{rSection}{Research Experience and Other Employments}
{\bf Research Assistant}{, New York State Division of the Budget} \hfill {05/2016 -- 08/2018}
\\
\end{rSection}
\bigskip  




%--------------------------------------------------------------------------------
%   Honors and Awards 
%-----------------------------------------------------------------------------------------------
\begin{rSection}{Scholarships, Honors, and Awards} 
{\bf Helen Horowitz Excellence in Teaching Award}{, University at Albany, SUNY} \hfill{2019}  \vspace{0.1cm}\\ 
{\bf Distinction in Preliminary Examinations}{, University at Albany, SUNY} \hfill{2015}  \vspace{0.1cm}\\
{\bf Graduate Assistantship}{, University at Albany, SUNY} \hfill{2014--2018}  
\vspace{0.1cm}\\
{\bf First Class Scholarship}{, Qingdao University} \hfill{2008, 2009} 
\vspace{0.1cm}\\
{\bf Outstanding Student Award}{, Qingdao University} \hfill{2008, 2009}  
\vspace{0.1cm}\\
\end{rSection}
\bigskip  




%----------------------------------------------------------------------------------------
%	Skilss
%----------------------------------------------------------------------------------------
\begin{rSection}{Skills}
{\bf Computer:} R, Stata, MATLAB, SAS, LaTeX 
\vspace{0.1cm} \\
{\bf Language:} Chinese Mandarin (Native), English (Fluent) 
\\
\end{rSection}
\bigskip  




%----------------------------------------------------------------------------------------
%	Personal Information
%----------------------------------------------------------------------------------------
\begin{rSection}{Personal Information }
%--copy and paste this region  if you need more--
{\bf Date of Birth:} July 7, 1989  
\vspace{0.1cm} \\
{\bf Gender:} Male  
\vspace{0.1cm} \\
{\bf Citizenship:} China  
\\
\end{rSection}
\bigskip




%----------------------------------------------------------------------------------------
%	References
%----------------------------------------------------------------------------------------
\begin{rSection}{References}
\begin{tabular}{lll}
	Professor Chun-Yu Ho  & Professor Betty Daniel  & Professor Zhongwen Liang \\ 
	Department of Economics &  Department of Economics & Department of Economics  \\
	University at Albany, SUNY & University at Albany, SUNY & University at Albany, SUNY  \\
	Hudson Building, Room 211A & Hudson Building, Room 243 & Hudson Building, Room 219 \\
	1400 Washington Avenue & 1400 Washington Avenue & 1400 Washington Avenue \\
	Albany, NY 12222 & Albany, NY 12222 & Albany, NY 12222 \\
	(+1) 518-442-4768 & (+1) 518-442-4747 & (+1) 518-442-4744 \\
	\href{mailto:cho@albany.edu}{\color{blue}{cho@albany.edu}} & \href{mailto:bdaniel@albany.edu}{\color{blue}{bdaniel@albany.edu}} & \href{mailto:zliang3@albany.edu}{\color{blue}{zliang3@albany.edu}}\\
\end{tabular}
\end{rSection}
\bigskip




%--------------------------------------------------------------------------------
%    Research Papers Abstracts
%-----------------------------------------------------------------------------------------------
\begin{rSection}{Abstracts of Research Papers}
	\textbf{``Monetary Policy Surprises and Interest Rates under China's Evolving Monetary Policy Framework''} (with Chun-Yu Ho) Revision Requested by \textit{\color{blue}{\underline{\textbf{Emerging Markets Review}}}}.\vspace{0.1cm}\\ 
	\textbf{Abstract:} The monetary policy in China has evolved considerably in the past two decades, increasingly moving from using quantity-based instruments and targets to using price-based ones. The monetary policy in China traditionally focused on quantity-based intermediate targets such as growth rates of monetary and credit aggregate (for example, M2). Since 2013, the central bank in China has introduced a range of lending facilities to develop an interest rate corridor, shifting the focus of the monetary policy toward price-based targets such as short-term market interest rates.  
	
	This paper assesses the effectiveness of monetary policy in China by examining the influence of monetary policy on market interest rates using a high-frequency event-study approach. We find that the effectiveness of price-based instruments in impacting market interest rates increases over time, and that price-based instruments are as effective as quantity instruments during the period since the completion of interest rates liberalization. Furthermore, central bank communications, an increasingly important aspect of monetary policy, affect medium- and long-term market interest rates. Our findings provide preliminary evidence on the effective transmission of the price-based monetary policy in China. \\
	
	\textbf{``Political Stability and Credibility of Currency Boards''} (with Shu Feng, Chun-Yu Ho, and Wai-Yip Alex Ho) Revision Requested by \textit{\color{blue}{\underline{\textbf{Journal of International Money and Finance}}}}.\vspace{0.1cm}\\
	\textbf{Abstract:} The currency board arrangement (CBA) is mainly characterized by a fixed nominal exchange rate against some anchor currency and the full backing of domestic central bank liabilities by foreign reserves. Under a CBA, the domestic monetary base is changed only through buying and selling the anchor currency at a fixed nominal exchange rate, removing discretion over monetary policy, and disciplining monetary authorities. Moreover, a CBA is usually codified in a law, which further increases the credibility of the system since any change would involve parliamentary or constitutional changes.  
	
	A pegged exchange rate regime is prone to speculative attacks since the possibility of adjustment under a currency peg can create an expectation of adjustment that is self-fulfilling. The main advantage of a CBA over a standard pegged exchange rate regime is the gain in the credibility of monetary policy. Do currency boards offer protection against self-fulfilling speculative attacks? This paper examines the credibility of currency boards of Argentina, Bulgaria, Estonia, Hong Kong, Latvia, and Lithuania. We estimate a Bayesian Markov switching model to analyze the role of economic fundamentals and self-fulfilling expectations in accounting for the credibility of the currency board. We find that the credibility of currency boards is subject to self-fulfilling runs. We also find that the credibility of currency boards positively relates to the political stability of adopting economies.\\
	
	\textbf{``Real Exchange Rate and Innovation: Firm-Level Evidence from China''} (with Chun-Yu Ho and Xiaoli Zhang) \vspace{0.1cm}\\
	\textbf{Abstract:} This paper examines how exchange rate movement affects firms' innovation activities using a panel dataset of Chinese manufacturing firms. We construct firm-specific effective real exchange rates (RER) to measure the exchange rate shocks faced by each firm according to its composition of trading partners. We find that a 10\% increase in effective RER (i.e., depreciation) increases innovation activities by about 0.2 percentage points. Our result is robust to 1) the inclusion of firm- and industry-specific control variables, firm-specific fixed effects, and year effects, 2) alternative measures of effective RER, and 3) alternative empirical specifications. We further show that a better export opportunity is the main channel through which depreciation in the exchange rate promotes innovation activities. A better export opportunity leads to higher revenue from exports, which in turn relaxes the financial constraint faced by firms to conduct innovation activities. \\
\end{rSection}
\bigskip \bigskip


\end{document}----------------------------

