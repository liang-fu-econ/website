%%%%%%%%%%%%%%%%%%%%%%%%%%%%%%%%%%%%%%%%%
% Important note:
% This template requires the resume.cls file to be in the same directory as the .tex file. 
% The resume.cls file provides the resume style used for structuring the document.
%%%%%%%%%%%%%%%%%%%%%%%%%%%%%%%%%%%%%%%%%






%----------------------------------------------------------------------------------------
%	PACKAGES AND OTHER DOCUMENT CONFIGURATIONS
%----------------------------------------------------------------------------------------

\documentclass{resume_liang} % Use the custom resume.cls style

\usepackage[left=0.75in, top=0.6in, right=0.75in, bottom=0.6in]{geometry} % Document margins
\usepackage[colorlinks=true, citecolor=blue]{hyperref} % Hypertext links in the document
% citecolor (green): Color for bibliographical citations in text.
\usepackage{tabu}

\newcommand{\tab}[1]{\hspace{.2667\textwidth}\rlap{#1}}
\newcommand{\itab}[1]{\hspace{0em}\rlap{#1}}




\name{Liang Fu} % Your name
\address{\small Last updated: Jun. 2021}
%\address{Hudson Building - Room 103} % Your address
%\address{1400 Washington Avenue, Albany, NY 12222} % Your secondary address (optional)
%\address{\href{mailto:lfu@albany.edu}{lfu@albany.edu} \\ (+1) 518-423-8555} %  Your phone number and your email






\begin{document}
	
%----------------------------------------------------------------------------------------
%	Contact Information
%----------------------------------------------------------------------------------------
\begin{tabu} to \linewidth {X[l]X[r]}
	%	 & \emph{Office:} (212) 998-8036 \\
	Hudson Building, Room 103     &  \href{mailto:lfu@albany.edu}{lfu@albany.edu}  \\
	1400 Washington Avenue        & +1 (518)912-9346\\
	Albany, NY 12222              & \href{https://liang-fu-econ.github.io/website/}{https://liang-fu-econ.github.io/website/}
\end{tabu}
\bigskip \bigskip




%----------------------------------------------------------------------------------------
%	Contact Information
%----------------------------------------------------------------------------------------
%\begin{rSection}{Contact Information}
%{\bf Email:} \href{mailto:lfu@albany.edu}{lfu@albany.edu}   \\
%{\bf Phone:} (+1) 518-423-8555  \\
%{\bf Office Address:} Hudson Building - Room 103, 1400 Washington Avenue, Albany, NY 12222 \\
%{\bf Home Address:} 101 Lincoln Avenue, Albany, NY 12206
%\end{rSection}
%\bigskip \bigskip





%----------------------------------------------------------------------------------------
%	Education 
%----------------------------------------------------------------------------------------

\begin{rSection}{Education}
%--copy and paste this region  if you need more--
Ph.D. Economics, {\bf University at Albany, SUNY}                     
  \begin{itemize}
  	\item[] \underline{Thesis Title}: Three essays in international economics
  	\item[] \underline{Expected Completion Date}: August 2021
  \end{itemize}
M.A. Economics, {\bf Central University of Finance and Economics}    \hfill {\em 2014} \\
B.S. Management, {\bf Qingdao University}                            \hfill {\em 2011} \\
\end{rSection}
\bigskip \bigskip






%----------------------------------------------------------------------------------------
%	Teaching and Research Fields
%----------------------------------------------------------------------------------------

\begin{rSection}{Research and Teaching Fields}
{\bf Primary fields:} International Economics, Monetary Economics \\ 
{\bf Secondary fields:} Chinese Economy  \\
\end{rSection}
\bigskip \bigskip






%--------------------------------------------------------------------------------
%    Research Papers
%-----------------------------------------------------------------------------------------------
\begin{rSection}{Research Papers}
\textbf{``Monetary Policy Surprises and Interest Rates under China‘s Evolving Monetary Policy Framework''} (with Chun-Yu Ho) [Under review]\\
Abstract: This paper assesses the effectiveness of monetary policy in China by examining the influence of monetary policy on market interest rates using an event-study approach. The monetary policy framework in China has evolved considerably in the past two decades, increasingly moving from using administrative and quantity instruments to price-based instruments. Overall, we find that price-based instruments are nearly as effective as quantity and administrative instruments in impacting market interest rates, especially since the completion of interest rate liberalization. Furthermore, central bank communications containing information on future monetary policy affect medium- and long-term market interest rates.  
\\
\\
\textbf{``Political Stability and Credibility of Currency Boards''} (with Shu Feng, Chun-Yu Ho and Wai-Yip Alex Ho) [Under review] \\
Abstract: This paper examines the credibility of currency boards of Argentina, Bulgaria, Estonia, Hong Kong, Latvia and Lithuania. We employ a Markov switching model to estimate the credibility of a currency board as the expected rate of depreciation induced by self-fulfilling behavior. The currency board of our sample countries are all subject to self-fulfilling behavior, which suggests that the credibility of a currency board determines the expected rate of depreciation. We also find evidence that the credibility of currency boards positively relates to the political stability of adopting economies.
\\
\\
`\textbf{`Real Exchange Rate and Innovation: Firm-Level Evidence from China''} (with Chun-Yu Ho and Xiaoli Zhang) \\
Abstract: This paper examines how exchange rate movement affects firms' innovation activities using a panel dataset of Chinese manufacturing firms. We construct firm-specific effective real exchange rate (RER) to measure the exchange rate shocks faced by each firm according to its composition of trading partners. We find that a 10\% increase in effective RER (i.e. depreciation) increase the share of new product sales in total sales by about 0.2 percentage points. Our results are robust to 1) the inclusion of firm- and industry-specific control variables, firm fixed effects and year effects; and 2) the use of alternative weighting in constructing effective RER and alternative estimation methods. We further show that a better export opportunity is the main channel through which a depreciation of exchange rate promotes innovation activities. A better export opportunity leads to a higher revenue from exports, which in turn, we argue that, relaxes the financial constraints faced by firms to conduct innovation activities.
\\
\\
\textbf{``Differential Regional Impact of U.S. Monetary Policy: Evidence from a Near VAR''} \\
Abstract: This paper finds that U.S. monetary policy had different effects across states during the 1979-2007 period by estimating a near vector autoregressive (VAR) model, in which each state's economy is affected by the national monetary policy as well as regional variables, while the common monetary policy responds only to aggregate macroeconomic variables. States in which (durable) manufacturing accounts for a larger share of gross state product tend to be more sensitive to monetary policy shocks. States containing a larger concentration of small firms are more responsive to monetary policy shocks, providing support for a broad credit channel of monetary policy transmission.
\\

\end{rSection}
\bigskip \bigskip




%--------------------------------------------------------------------------------
%    Research Papers in Progress
%-----------------------------------------------------------------------------------------------
\begin{rSection}{Research Papers in Progress}
\textbf{``The Impact of China's Monetary Policy on Exchange Rates: Evidence from High-frequency Data''}\\
Abstract: This paper examines the behavior of the Chinese renminbi/U.S. dollar exchange rate in reaction to China's monetary policy actions and statements, using an event study approach coupled with intra-day high-frequency data.

\end{rSection}
\bigskip \bigskip





%----------------------------------------------------------------------------------------
%    Teaching Experience
%----------------------------------------------------------------------------------------
\begin{rSection}{Teaching Experience}
{\bf Instructor}{, University at Albany, SNUY} \hfill {\em 08/2018 - current}
 \begin{itemize}
     \item[] Money and Banking (Undergraduate): Summer 2018, Fall 2018, Fall 2019, Spring 2020
     \item[] Intermediate Macroeconomics (Undergraduate): Spring 2019, Fall 2020, Spring 2021
 \end{itemize} 
{\bf Teaching Assistant}{, University at Albany, SUNY} \hfill {\em 09/2014 - 05/2016}
   \begin{itemize}
     \item[] Microeconomics II (PhD): Spring 2016
     \item[] Public Microeconomics (Undergraduate): Spring 2016
     \item[] Applied Econometrics (Undergraduate): Fall 2015
     \item[] Economics of Labor (Undergraduate): Fall 2014, Spring 2015
   \end{itemize} 
{\bf Teaching Assistant}{, Central University of Finance and Economics} \hfill {\em 02/2012 - 01/2013}
\begin{itemize}
	\item[] Production and Consumption Theory (Undergraduate): Fall 2012
	\item[] Intermediate Microeconomics (Undergraduate): Spring 2012 
\end{itemize} 
\end{rSection}
\bigskip \bigskip





%----------------------------------------------------------------------------------------
%	Research Experience 
%----------------------------------------------------------------------------------------
\begin{rSection}{Research Experience and other Employments}
	{\bf Research Assistant}{, New York State Division of the Budget} \hfill {\em 05/2016 - 08/2018}
\end{rSection}
\bigskip \bigskip






%--------------------------------------------------------------------------------
%   Honors and Awards 
%-----------------------------------------------------------------------------------------------
\begin{rSection}{Scholarships, Honors, and Awards} 
{\bf Helen Horowitz Excellence in Teaching Award}{, University at Albany, SUNY} \hfill{\em 2019}  \\ 
{\bf Distinction in Preliminary Examinations}{, University at Albany, SUNY} \hfill{\em 2015}  \\
{\bf Graduate Assistantship}{, University at Albany, SUNY} \hfill{\em 2014 - 2016}  \\
\end{rSection}
\bigskip \bigskip





%----------------------------------------------------------------------------------------
%	Skilss
%----------------------------------------------------------------------------------------
\begin{rSection}{Skills}
{\bf Computer:} R, Stata, MATLAB, SAS, LaTeX
\\
{\bf Language:} Chinese Mandarin (Native), English (Fluent) 
\end{rSection}
\bigskip \bigskip








%----------------------------------------------------------------------------------------
%	Personal Information
%----------------------------------------------------------------------------------------

\begin{rSection}{Personal Information }
%--copy and paste this region  if you need more--
{\bf Date of Birth:} July 7, 1989  \\
{\bf Gender:} Male  \\
{\bf Citizenship:} China
\\
\end{rSection}




\end{document}----------------------------

